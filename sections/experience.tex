 \documentclass[]{article}
\usepackage{TLCresume}

\begin{document}

\vspace{1ex}

\subsection{{Software Engineer (Temporary Employee)\hfill{\small{March 2022 - Today}}}}
\subtext{Qualcomm \hfill{\small{San Diego, CA}}}
\vspace{1.5ex}
% - To improve the automotive chipsets of Qualcomm, I have written over 150 APIs for a framework tool designed to enable in-vehicle features, built by a large team of developers,  using virtual machines, git, QNX, and ADB. \\
% - I implemented graphics features supporting \textbf{OpenCL}, \textbf{OpenGL} and \textbf{EGL} by creating kernels, contexts and command queues. I also enabled support of the audio APIs \textbf{ALSA} for Linux by designing a capture-playback loop. 
% I am currently focusing on Qualcomm internal camera APIs, all using C++ and Python. \\
\begin{itemize}
    \item Developed around 200 APIs with Python and C++ for a framework used to support numerous libraries across 3 generations of automotive chipsets.
    \item Enabled graphical features using \textbf{OpenCL, OpenGL, and EGL}, managing kernels, context, and command queues; performed validation on Qualcomm's internal camera APIs for the latest chipset firmware builds.
    \item Developed an audio capture and playback tool using \textbf{ALSA} for Linux-based virtual machines.
    % \item Involved in an agile development environment using \textbf{Jira} and version control with \textbf{git}.
\end{itemize}

\vspace{2ex}

\subsection{Machine Learning Engineer Intern \hfill{\small{Jun 2021 - Aug 2021}}}

\subtext{Kapaix Ltd \hfill{\small{London, England (Remote)}}}
\vspace{1.5ex}

\begin{itemize}
    \item Designed neural network models for anomaly detection purposes, analyzing discrepancies in frequencies and amplitudes of data points in time series to assess the quality of a database for a Big Data Management company.
    \item Preprocessed the dataset by building histograms with variable time frames, using \textbf{PCA} and \textbf{k-means clustering} as the first analysis tool.
    \item Constructed two ML architectures: a classification model and an autoencoder model, using dense and convolutional layers with \textbf{Python: \textit{Keras - TensorFlow - Pandas}}.
\end{itemize}


\end{document}